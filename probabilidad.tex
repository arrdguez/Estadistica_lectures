\documentclass[11pt]{beamer}
%\usefonttheme{professionalfonts}
\usefonttheme{serif}

\usepackage[utf8]{inputenc}
\usepackage[T1]{fontenc}
\usepackage{lmodern}
\usepackage[spanish]{babel}
\usepackage{amsmath}
\usepackage{amssymb}
\usetheme{EastLansing}
\usepackage{graphicx}
\usepackage[]{xcolor}
\usepackage{tikz}
\usetikzlibrary{shapes.geometric, arrows}

\author{Dr. Alejandro Rodriguez}
\title{Probabilidad y Estad\'istica\\Probabilidad}
\subtitle{Universidad Tecnol\'ogica Iz\'ucar de Matamoros\\UTIM }


\begin{document}

    \begin{frame}[plain]
        \maketitle
    \end{frame}

    \begin{frame}{Temas de Probabilidad}
      \begin{itemize}
          \item Conjuntos
          \item Probabilidad Básica y Condicional
          \item Distribuciones Discretas de Probabilidad
          \item Distribuciones Continuas de Probabilidad
          \item Distribuciones Muestrales
      \end{itemize}
    \end{frame}
    \begin{frame}{Probabilidad}
       \begin{block}{Probabilidad}
           El término \textbf{probabilidad} se refiere al estudio de azar y la incertidumbre en cualquier
           situación en la cual varios posibles sucesos pueden ocurrir.
       \end{block}
       \pause
       En palabras simples, fenómenos aleatorios son los que pueden dar lugar a varios resultados, sin que pueda ser posible enunciar con certeza cuál de éstos va a ser observado en la realización del experimento.
    \end{frame}



    \section*{Conjuntos}
      \begin{frame}{Espacio muestral}
          \begin{block}{title}
              Definir los conceptos y notación de conjuntos:
              -Universo
              -Vacío
              -Subconjunto

              Describir el proceso de construcción del diagrama de Venn Euler.

              Explicar las operaciones entre conjuntos:
              - Unión
              - Intersección
              - Complemento
              - Diferencia
          \end{block}
      \end{frame}

      \begin{frame}{Conjuntos. Espacio muestral}
          \begin{block}{Espacio muestral}
              El \textbf{espacio muestral} de un experimento denotado por $E$ , es el conjunto de todos los posibles resultados de dicho experimento.
          \end{block}
          Ejemplos:
          \begin{enumerate}[<+->]
              \item El espacio muestral asociado a lanzar un dado, E = {1,2,3,4,5,6}
              \item El espacio asociado a preguntar a un cliente si le gusta o no nuestro producto es E ={S, N} (S – sí; N – no)
              \item El espacio asociado a indagar si 3 clientes que entraron a una tienda compraron un producto es E = {SSS, SSN, SNS, NSS, SNN, NSN, NNS, NNN}.
          \end{enumerate}
      \end{frame}


    \section*{Probabilidad Básica y Condicional}
      \begin{frame}{title}
        \begin{block}{title}
            content...
        \end{block}
      \end{frame}

    \section*{Distribuciones Discretas de Probabilidad}
    \begin{frame}{title}
        content...
    \end{frame}



    \section*{Distribuciones Continuas de Probabilidad}
    \begin{frame}{title}
        \begin{block}{title}
            content...
        \end{block}
    \end{frame}



    \section*{Distribuciones Muestrales}
    \begin{frame}{title}
        \begin{block}{title}
            content...
        \end{block}
    \end{frame}


\end{document}