\documentclass[11pt,letterpaper]{report}
\usepackage[utf8]{inputenc}
\usepackage[T1]{fontenc}
\usepackage[spanish]{babel}
\usepackage{amsmath}
\usepackage{amsfonts}
\usepackage{amssymb}
\usepackage{makeidx}
\usepackage{makeidx}
\usepackage{graphicx}
\usepackage{mdframed}
\usepackage{xcolor}
\usepackage[left=2.00cm, right=2.00cm]{geometry}
\author{Dr. Alejandro Rodr\'iguez}
\title{Clases Practicas}
\date{Universidad Tecnol\'ogica de Iz\'ucar de Matamoros}


\newenvironment{problem}[2][Ejercicio]
{ \begin{mdframed}[] \textbf{#1 #2} \\}
    {  \end{mdframed}}




\begin{document}
  \maketitle
  \newpage

  \chapter*{Nota preliminar}
    Los ejercicios a continuación tiene como propósito ayudarle al estudiante a reconocer debilidades y fortalezas propias en el conocimiento de la materia. Se le recomienda a los estudiantes investigar de forma autodidacta ante las dudas, de modo que desarrolle la capacidad de búsqueda y solución de los problemas planteados.

    Algunos términos y conceptos difieren entre autores y documentos disponibles en la red o en la biblioteca, por favor sea sensible y comprensible con esto. Ante cualquier duda no dude en preguntar.

    Cualquier error encontrada en este texto, por favor hágalo saber al profesor de la asignatura.
    PD: También se aceptan sugerencias.


  \chapter{Estad\'istica Descriptiva}
    \section{Introducción a la estadística}

      \subsection*{Ejercicio 1}
        Con sus propias palabras defina: Estadística, Estadística Descriptiva y Datos en la estadística.
      \subsection*{Ejercicio 2}
        ¿Que son las variables estadísticas?.
      \subsection*{Ejercicio 3}
         Dada los siguientes datos, clasifique a que tipo de variable estadística pertenece.
        \begin{itemize}
      	  \item Masculino, Femenino
          \item 3.45Kg, 45.6Kg, 105.0Kg
          \item -2, 1000, -3045, 0
          \item Soltero, Casado, Viudo, Divorciado, Unión Libre
          \item Número de Hijos
          \item Estado Civil
          \item Números enteros positivos
          \item Números enteros negativos
          \item Verde, Amarillo, Rojo
          \item ¿Está vacunado?, ¿Sí o No?
        \end{itemize}
      \subsection*{Ejercicio 4}
        Confeccione una tabla de datos que contenga cada uno de los tipos de variables que usted desee.


    \section{Población, muestra y muestreo}
      \subsection*{Ejercicio 6}
        Defina con sus palabras: Universo o Población, Muestra y Elemento.
      \subsection*{Ejercicio 7}
        De los universos siguientes, ¿cuáles están definidos rigurosamente y cuáles no?¿Por qué?
        \begin{enumerate}
      	  \item [a.] Habitantes de la ciudad de Puebla mayores de 18 años. Marzo de 1993.
      	  \item [b.] Estudiantes de Ingeniería de 2000.
      	  \item [c.] Obrero de planta permanente de la fábrica de autos Volkswagen.
      	  \item [d.] Los números primos menores que 20.
      	  \item [e.] Viviendas con más de dos recamaras en la Ciudad Izúcar de Matamoros.
        \end{enumerate}
      \subsection*{Ejercicio 8}
        Dé una posible muestra de tamaño 4 de cada una de las siguientes poblaciones.
        \begin{itemize}
            \item[a.] Todo el personal de una clínica.
            \item[b.] Todos los ríos de México.
            \item[c.] Todos los estudiantes en su colegio o universidad.
            \item[d.] Todas las calificaciones promedio de los estudiantes en su
            colegio o universidad.
        \end{itemize}

      \subsection*{Ejercicio 9}
        Se desea tomar una muestra aleatoria estratificada de las personas mayores de edad de un municipio, cuyos estratos son los siguientes intervalos de edades, en años: de 18 a 30, de 31 a 45, de 46 a 60 y mayores de 60. En el primer intervalo hay 7500 personas, en el segundo hay 8400, en el tercero 5700 y en el cuarto 3000. Calcule el tamaño de la muestra total y su composición, sabiendo que el muestreo se hace de forma proporcional y se han elegido al azar 375 personas del primer estrato.
      \subsection*{Ejercicio 10}
        En un pueblo habitan 700 hombres adultos, 800 mujeres adultas y 500 menores. De él se quiere seleccionar una muestra de 80 personas, utilizando, para ello, muestreo estratificado y se  tomaran de forma proporcional proporcional. ¿Cuál será la composición que debe tener dicha muestra?
      \subsection*{Ejercicio 11}
        Una ganadería tiene 3000 vacas. Se quiere extraer una muestra de 120. Explica cómo se obtiene
        dicha muestra:
        a) Mediante muestreo aleatorio simple.
        b) Mediante muestreo aleatorio sistemático.
    \section{Distribución de frecuencias y su representación gráfica}
      \subsection*{Ejercicio 12}
        Dada la distribución siguiente, constrúyase una tabla estadística en la que aparezcan las frecuencias absolutas, las frecuencias relativas y las frecuencias acumuladas relativas crecientes:
        \begin{table}[!h]
          \centering
          \begin{tabular}{|c|cccccc|}
              \hline
              $x_i$ & 1 & 2 & 3 & 4 & 5 & 6  \\
              \hline
              $n_i$ & 5 & 7 & 9 & 6 & 7 & 6  \\
              \hline
          \end{tabular}
        \end{table}
      \subsection*{Ejercicio 13}
        Los datos que se dan a continuación corresponden a los pesos en Kg. de ochenta personas:\\

        (a) Obténgase una distribución de datos en intervalos de amplitud 5, siendo el primer intervalo [50; 55].

        (b) Calcúlese el porcentaje de personas de peso menor que 65 Kg.

        (c) ¿Cuántas personas tienen peso mayor o igual que 70 Kg. pero menor que 85? \\

        60; 66; 77; 70; 66; 68; 57; 70; 66; 52; 75; 65; 69; 71; 58; 66; 67; 74; 61; 63; 69; 80; 59; 66; 70; 67; 78; 75; 64; 71; 81; 62; 64; 69; 68; 72; 83; 56; 65; 74; 67; 54; 65; 65; 69; 61; 67; 73; 57; 62; 67; 68; 63; 67; 71; 68; 76;
        61; 62; 63; 76; 61; 67; 67; 64; 72; 64; 73; 79; 58; 67; 71; 68; 59; 69; 70; 66; 62; 63; 66;
      \subsection*{Ejercicio 14}
        Las edades de los empleados de una determinada empresa son las que aparecen en la siguiente tabla:

        \begin{table}[!h]
          \centering
          \begin{tabular}{|c|c|}
              \hline
              Edad & N$^o$ empleados   \\
              \hline
              Menos de 25 & 22 \\
              Menos de 35 & 70 \\
              Menos de 45 & 121\\
              Menos de 55 & 157\\
              Menos de 65 & 184\\
              \hline
          \end{tabular}
        \end{table}

        Sabiendo que el empleado más joven tiene 18 años, escríbase la distribución de frecuencias acumuladas decrecientes (o «más de»).
      \subsection*{Ejercicio 15}
        Las temperaturas medias registradas durante el mes de mayo en Madrid, en grados centígrados, están dadas por la siguiente tabla:
        \begin{table}[!h]
            \centering
            \begin{tabular}{|c|c|c|c|c|c|c|c|c|c|c|}
                \hline
                Temperatura & 13 &14& 15& 16& 17 &18& 19& 20& 21& 22  \\
                \hline
               N.$^o$ de días &1 &1 &2 &3 &6 &8 &4 &3 &2& 1   \\
                \hline
            \end{tabular}
        \end{table}
      \subsection*{Ejercicio 16}
         Dada la distribución de frecuencias:
         \begin{table}[!h]
             \centering
             \begin{tabular}{|c|c|}
                 \hline
                 $x_i$ & $n_i$ \\
                 \hline
                 1  & 9   \\
                 \hline
                 2  & 22  \\
                 \hline
                 3  & 13  \\
                 \hline
                 4  & 23  \\
                 \hline
                 5  & 8   \\
                 \hline
                 6  & 25  \\
                 \hline
             \end{tabular}
         \end{table}
         \\
         (a) Constrúyase una tabla en la que aparezcan frecuencias absolutas, frecuencias relativas, frecuencias acumuladas absolutas crecientes (o «menos de») y decrecientes (o «más de»).\\
         (b) Represéntese mediante un diagrama de barras la distribución dada y su correspondiente polígono de frecuencias.\\
         (c) Obténgase el polígono de frecuencias absolutas acumuladas crecientes y decrecientes.\\
      \subsection*{Ejercicio 17}
        Represéntese gráficamente la siguiente distribución de frecuencias:
        \begin{table}[!h]
            \centering
            \begin{tabular}{|c|c|}
                \hline
                $L_{i-1}-L_i$  & $n_i$\\
                \hline
                0-10& 22\\
                \hline
                10-20& 26\\
                \hline
                20-30& 92\\
                \hline
                30-40& 86\\
                \hline
                40-50& 74\\
                \hline
                50-60& 27\\
                \hline
                60-70& 12\\
                \hline
            \end{tabular}
        \end{table}
      \subsection*{Ejercicio 18}
        Encuestados cincuenta matrimonios respecto a su número de hijos, se obtuvieron los siguientes datos:\\

        2; 4; 2; 3; 1; 2; 4; 2; 3; 0; 2; 2; 2; 3; 2; 6; 2; 3; 2; 2; 3; 2; 3; 3; 4; 1; 3; 3; 4; 5; 2; 0; 3; 2; 1; 2; 3; 2; 2; 3; 1; 4; 2; 3; 2; 4; 3; 3; 2\\

        Constrúyase una tabla estadística que represente dichos datos
      \subsection*{Ejercicio 19}



    \section{Medidas de tendencia central, localización y dispersión}
      \subsection*{Ejercicio 19}
        Calculo de la media aritmética, la mediana y la moda. Se analizó el IVA que se aplica, en diversos países europeos, a la compra de obras de arte. Los resultados obtenidos fueron los siguientes:
        \begin{table}[!h]
            \begin{tabular}{|c|c|}
                \hline
                PAIS& \\
                \hline
                España &0,16\\
                Italia &0,20\\
                Bélgica& 0,06\\
                Holanda& 0,06\\
                Alemania& 0,07\\
                Portugal& 0,17\\
                Luxemburgo& 0,06\\
                Finlandia& 0,22\\
                \hline
            \end{tabular}
        \end{table}

  \chapter{Probabilidad}
  \chapter{Estad\'istica Inferencial}
\end{document}