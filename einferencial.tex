\documentclass[11pt]{beamer}
\usepackage[utf8]{inputenc}
\usepackage[T1]{fontenc}
\usepackage{lmodern}
\usepackage[english,spanish]{babel}
\usetheme{CambridgeUS}
\begin{document}
    \author{Dr. Alejandro Rodr\'iguez}
    \title{Estadística Inferencial}
    %\subtitle{}
    %\logo{}
    %\institute{}
    %\date{}
    %\subject{}
    %\setbeamercovered{transparent}
    %\setbeamertemplate{navigation symbols}{}
    \begin{frame}[plain]
        \maketitle
    \end{frame}

    \section{Introducción a la estadística inferencial.}
      \begin{frame}{Estadística inferencial}
        \begin{block}{Estadística inferencial}
            Se llama estadística inferencial o inferencia estadística a la rama de la Estadística encargada de hacer \textbf{deducciones}, es decir, \textbf{inferir propiedades}, \textbf{conclusiones} y \textbf{tendencias}, a partir de una muestra del conjunto. \textbf{Su papel es interpretar, hacer proyecciones y comparaciones.}
        \end{block}
      \end{frame}


      \begin{frame}{Estadística inferencial}
          \begin{block}{Estadística inferencial}
              \begin{itemize}
                  \item Estimación estadística
                  \item Prueba de
Hipótesis
                  \item Regresión Lineal y
Correlación
                  \item Diseño de
experimentos
              \end{itemize}
          \end{block}
      \end{frame}

      \subsection{Estimación estadística}
        \begin{frame}{Estimación estadística}
            \begin{block}{Estimación estadística}
                La estimación es la determinación de un elemento o factor. Esto, usualmente tomando como referencia una base o conjunto de datos.
            \end{block}
            \pause
            La estimación es un cálculo que se realiza a partir de la evaluación estadística. Dicho estudio suele efectuarse sobre una muestra y no sobre toda la población objetivo.
            Sea $X_1, X_2, X_3, \ldots , X_n $ una muestra aleatoria de una distribución, un estimador es un estadístico $\hat{\theta}=T(X_1, X_2, X_3, \ldots , X_n)$ que sirve para estimar el valor $\theta$.

            \pause
            Nos sirve para calcular indicadores estadísticos como la \textbf{media}, la\textbf{ mediana}, \textbf{moda} y \textbf{desviación estándar}.
        \end{frame}

        \begin{frame}{Estimación estadística. Puntual}
            \begin{block}{Estimación estadística Puntual}
                La estimación puntual consiste en encontrar un valor para $\theta$, denotado por  $\hat {\theta }$. Es decir seleccionar aquel estadístico que mejor nos permita describir la muestra.

                \pause
                \textbf{Ejemplo:} Por ejemplo, si se pretende estimar la talla media de un determinado grupo de individuos, puede extraerse una muestra y ofrecer como estimación puntual la talla media de los individuos.
            \end{block}

        \end{frame}









\end{document}